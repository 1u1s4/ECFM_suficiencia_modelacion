% Options for packages loaded elsewhere
\PassOptionsToPackage{unicode}{hyperref}
\PassOptionsToPackage{hyphens}{url}
%
\documentclass[
]{article}
\usepackage{lmodern}
\usepackage{amssymb,amsmath}
\usepackage{ifxetex,ifluatex}
\ifnum 0\ifxetex 1\fi\ifluatex 1\fi=0 % if pdftex
  \usepackage[T1]{fontenc}
  \usepackage[utf8]{inputenc}
  \usepackage{textcomp} % provide euro and other symbols
\else % if luatex or xetex
  \usepackage{unicode-math}
  \defaultfontfeatures{Scale=MatchLowercase}
  \defaultfontfeatures[\rmfamily]{Ligatures=TeX,Scale=1}
\fi
% Use upquote if available, for straight quotes in verbatim environments
\IfFileExists{upquote.sty}{\usepackage{upquote}}{}
\IfFileExists{microtype.sty}{% use microtype if available
  \usepackage[]{microtype}
  \UseMicrotypeSet[protrusion]{basicmath} % disable protrusion for tt fonts
}{}
\makeatletter
\@ifundefined{KOMAClassName}{% if non-KOMA class
  \IfFileExists{parskip.sty}{%
    \usepackage{parskip}
  }{% else
    \setlength{\parindent}{0pt}
    \setlength{\parskip}{6pt plus 2pt minus 1pt}}
}{% if KOMA class
  \KOMAoptions{parskip=half}}
\makeatother
\usepackage{xcolor}
\IfFileExists{xurl.sty}{\usepackage{xurl}}{} % add URL line breaks if available
\IfFileExists{bookmark.sty}{\usepackage{bookmark}}{\usepackage{hyperref}}
\hypersetup{
  hidelinks,
  pdfcreator={LaTeX via pandoc}}
\urlstyle{same} % disable monospaced font for URLs
\usepackage{color}
\usepackage{fancyvrb}
\newcommand{\VerbBar}{|}
\newcommand{\VERB}{\Verb[commandchars=\\\{\}]}
\DefineVerbatimEnvironment{Highlighting}{Verbatim}{commandchars=\\\{\}}
% Add ',fontsize=\small' for more characters per line
\newenvironment{Shaded}{}{}
\newcommand{\AlertTok}[1]{\textcolor[rgb]{1.00,0.00,0.00}{\textbf{#1}}}
\newcommand{\AnnotationTok}[1]{\textcolor[rgb]{0.38,0.63,0.69}{\textbf{\textit{#1}}}}
\newcommand{\AttributeTok}[1]{\textcolor[rgb]{0.49,0.56,0.16}{#1}}
\newcommand{\BaseNTok}[1]{\textcolor[rgb]{0.25,0.63,0.44}{#1}}
\newcommand{\BuiltInTok}[1]{#1}
\newcommand{\CharTok}[1]{\textcolor[rgb]{0.25,0.44,0.63}{#1}}
\newcommand{\CommentTok}[1]{\textcolor[rgb]{0.38,0.63,0.69}{\textit{#1}}}
\newcommand{\CommentVarTok}[1]{\textcolor[rgb]{0.38,0.63,0.69}{\textbf{\textit{#1}}}}
\newcommand{\ConstantTok}[1]{\textcolor[rgb]{0.53,0.00,0.00}{#1}}
\newcommand{\ControlFlowTok}[1]{\textcolor[rgb]{0.00,0.44,0.13}{\textbf{#1}}}
\newcommand{\DataTypeTok}[1]{\textcolor[rgb]{0.56,0.13,0.00}{#1}}
\newcommand{\DecValTok}[1]{\textcolor[rgb]{0.25,0.63,0.44}{#1}}
\newcommand{\DocumentationTok}[1]{\textcolor[rgb]{0.73,0.13,0.13}{\textit{#1}}}
\newcommand{\ErrorTok}[1]{\textcolor[rgb]{1.00,0.00,0.00}{\textbf{#1}}}
\newcommand{\ExtensionTok}[1]{#1}
\newcommand{\FloatTok}[1]{\textcolor[rgb]{0.25,0.63,0.44}{#1}}
\newcommand{\FunctionTok}[1]{\textcolor[rgb]{0.02,0.16,0.49}{#1}}
\newcommand{\ImportTok}[1]{#1}
\newcommand{\InformationTok}[1]{\textcolor[rgb]{0.38,0.63,0.69}{\textbf{\textit{#1}}}}
\newcommand{\KeywordTok}[1]{\textcolor[rgb]{0.00,0.44,0.13}{\textbf{#1}}}
\newcommand{\NormalTok}[1]{#1}
\newcommand{\OperatorTok}[1]{\textcolor[rgb]{0.40,0.40,0.40}{#1}}
\newcommand{\OtherTok}[1]{\textcolor[rgb]{0.00,0.44,0.13}{#1}}
\newcommand{\PreprocessorTok}[1]{\textcolor[rgb]{0.74,0.48,0.00}{#1}}
\newcommand{\RegionMarkerTok}[1]{#1}
\newcommand{\SpecialCharTok}[1]{\textcolor[rgb]{0.25,0.44,0.63}{#1}}
\newcommand{\SpecialStringTok}[1]{\textcolor[rgb]{0.73,0.40,0.53}{#1}}
\newcommand{\StringTok}[1]{\textcolor[rgb]{0.25,0.44,0.63}{#1}}
\newcommand{\VariableTok}[1]{\textcolor[rgb]{0.10,0.09,0.49}{#1}}
\newcommand{\VerbatimStringTok}[1]{\textcolor[rgb]{0.25,0.44,0.63}{#1}}
\newcommand{\WarningTok}[1]{\textcolor[rgb]{0.38,0.63,0.69}{\textbf{\textit{#1}}}}
\setlength{\emergencystretch}{3em} % prevent overfull lines
\providecommand{\tightlist}{%
  \setlength{\itemsep}{0pt}\setlength{\parskip}{0pt}}
\setcounter{secnumdepth}{-\maxdimen} % remove section numbering

\author{}
\date{}

\begin{document}

\hypertarget{paquetes-a-utilizar}{%
\section{Paquetes a utilizar}\label{paquetes-a-utilizar}}

\begin{Shaded}
\begin{Highlighting}[]
\ImportTok{import}\NormalTok{ sklearn }\ImportTok{as}\NormalTok{ skl}
\ImportTok{import}\NormalTok{ pandas }\ImportTok{as}\NormalTok{ pd}
\ImportTok{import}\NormalTok{ numpy }\ImportTok{as}\NormalTok{ np}
\ImportTok{import}\NormalTok{ xgboost }\ImportTok{as}\NormalTok{ xgb}
\ImportTok{from}\NormalTok{ boruta }\ImportTok{import}\NormalTok{ BorutaPy}
\end{Highlighting}
\end{Shaded}

\hypertarget{base-de-datos}{%
\section{Base de datos}\label{base-de-datos}}

\begin{Shaded}
\begin{Highlighting}[]
\NormalTok{df }\OperatorTok{=}\NormalTok{ pd.read\_csv(}\StringTok{"data/wisconsin\_breast\_cancer\_dataset.csv"}\NormalTok{)}
\NormalTok{df }\OperatorTok{=}\NormalTok{ df.dropna()}
\NormalTok{df }\OperatorTok{=}\NormalTok{ df.rename(columns}\OperatorTok{=}\NormalTok{\{}\StringTok{\textquotesingle{}diagnosis\textquotesingle{}}\NormalTok{:}\StringTok{\textquotesingle{}Label\textquotesingle{}}\NormalTok{\})}
\NormalTok{df[}\StringTok{\textquotesingle{}Label\textquotesingle{}}\NormalTok{].value\_counts()}
\end{Highlighting}
\end{Shaded}

\begin{Shaded}
\begin{Highlighting}[]
\NormalTok{df.describe().T}
\end{Highlighting}
\end{Shaded}

\begin{Shaded}
\begin{Highlighting}[]
\NormalTok{df.dtypes}
\end{Highlighting}
\end{Shaded}

\hypertarget{variable-dependiente-que-debe-predecirse-y-codificaciuxf3n-de-datos-categuxf3ricos}{%
\section{Variable dependiente que debe predecirse y codificación de
datos
categóricos}\label{variable-dependiente-que-debe-predecirse-y-codificaciuxf3n-de-datos-categuxf3ricos}}

\begin{Shaded}
\begin{Highlighting}[]
\NormalTok{y }\OperatorTok{=}\NormalTok{ df[}\StringTok{"Label"}\NormalTok{].values}
\NormalTok{Y }\OperatorTok{=}\NormalTok{ skl.preprocessing.LabelEncoder().fit\_transform(y)}
\end{Highlighting}
\end{Shaded}

\hypertarget{definir-x-normalizar-valores-y-definir-variables-independientes}{%
\section{Definir x, normalizar valores y definir variables
independientes}\label{definir-x-normalizar-valores-y-definir-variables-independientes}}

\begin{Shaded}
\begin{Highlighting}[]
\NormalTok{X }\OperatorTok{=}\NormalTok{ df.drop(labels }\OperatorTok{=}\NormalTok{ [}\StringTok{"Label"}\NormalTok{, }\StringTok{"id"}\NormalTok{], axis}\OperatorTok{=}\DecValTok{1}\NormalTok{) }
\NormalTok{nombres\_de\_funciones }\OperatorTok{=}\NormalTok{ np.array(X.columns)}
\NormalTok{scaler }\OperatorTok{=}\NormalTok{ skl.preprocessing.StandardScaler()}
\NormalTok{scaler.fit(X)}
\NormalTok{X }\OperatorTok{=}\NormalTok{ scaler.transform(X)}
\end{Highlighting}
\end{Shaded}

\hypertarget{train-and-test-para-verificar-la-precisiuxf3n-despuuxe9s-de-ajustar-el-modelo}{%
\section{Train and test para verificar la precisión después de ajustar
el
modelo}\label{train-and-test-para-verificar-la-precisiuxf3n-despuuxe9s-de-ajustar-el-modelo}}

\begin{Shaded}
\begin{Highlighting}[]
\NormalTok{X\_train, X\_test, y\_train, y\_test }\OperatorTok{=}\NormalTok{ skl.model\_selection.train\_test\_split(X, Y, test\_size}\OperatorTok{=}\FloatTok{0.25}\NormalTok{, random\_state}\OperatorTok{=}\DecValTok{42}\NormalTok{)}
\end{Highlighting}
\end{Shaded}

\hypertarget{xgboost-para-ser-utilizado-por-boruta}{%
\section{XGBOOST para ser utilizado por
Boruta}\label{xgboost-para-ser-utilizado-por-boruta}}

\begin{Shaded}
\begin{Highlighting}[]
\NormalTok{modelo }\OperatorTok{=}\NormalTok{ xgb.XGBClassifier()}
\end{Highlighting}
\end{Shaded}

\begin{itemize}
\tightlist
\item
  Crear funciones de sombra: funciones aleatorias y valores aleatorios
  en columnas
\item
  Entrenar Random Forest / XGBoost y calcular la importancia de la
  característica a través de la disminución media de la impureza
\item
  Comprobar si las características reales tienen mayor importancia en
  comparación con las características de sombra
\item
  Repetir esto para cada iteración
\item
  Si la función original funcionó mejor, marcarla como importante
\end{itemize}

\begin{Shaded}
\begin{Highlighting}[]
\CommentTok{\# definir el método de selección de características de Boruta}
\NormalTok{selector }\OperatorTok{=}\NormalTok{ BorutaPy(modelo, n\_estimators}\OperatorTok{=}\StringTok{\textquotesingle{}auto\textquotesingle{}}\NormalTok{, verbose}\OperatorTok{=}\DecValTok{2}\NormalTok{, random\_state}\OperatorTok{=}\DecValTok{1}\NormalTok{)}
\CommentTok{\# encontrar todas las características relevantes}
\NormalTok{selector.fit(X\_train, y\_train)}
\CommentTok{\# llamar a transform() en X para filtrarlo a las características seleccionadas}
\NormalTok{X\_filtered }\OperatorTok{=}\NormalTok{ selector.transform(X\_train)  }\CommentTok{\# Aplicar selección de características y devolver datos transformados}
\end{Highlighting}
\end{Shaded}

\hypertarget{zip-nombres-de-caracteruxedsticas-rangos-y-decisiones}{%
\section{zip nombres de características, rangos y
decisiones}\label{zip-nombres-de-caracteruxedsticas-rangos-y-decisiones}}

\begin{Shaded}
\begin{Highlighting}[]
\NormalTok{feature\_ranks }\OperatorTok{=} \BuiltInTok{list}\NormalTok{(}\BuiltInTok{zip}\NormalTok{(nombres\_de\_funciones, }
\NormalTok{                         selector.ranking\_, }
\NormalTok{                         selector.support\_))}
\end{Highlighting}
\end{Shaded}

\hypertarget{resultados}{%
\section{Resultados}\label{resultados}}

\begin{Shaded}
\begin{Highlighting}[]
\ControlFlowTok{for}\NormalTok{ feat }\KeywordTok{in}\NormalTok{ feature\_ranks:}
    \BuiltInTok{print}\NormalTok{(}\StringTok{\textquotesingle{}Feature: }\SpecialCharTok{\{:\textless{}30\}}\StringTok{ Rank: }\SpecialCharTok{\{\}}\StringTok{,  Keep: }\SpecialCharTok{\{\}}\StringTok{\textquotesingle{}}\NormalTok{.}\BuiltInTok{format}\NormalTok{(feat[}\DecValTok{0}\NormalTok{], feat[}\DecValTok{1}\NormalTok{], feat[}\DecValTok{2}\NormalTok{]))}
\end{Highlighting}
\end{Shaded}

\begin{Shaded}
\begin{Highlighting}[]
\CommentTok{\# Ahora usar el subconjunto de funciones para ajustar el modelo XGBoost en los datos de entrenamiento}
\NormalTok{xgb\_model }\OperatorTok{=}\NormalTok{ xgb.XGBClassifier()}
\NormalTok{xgb\_model.fit(X\_filtered, y\_train)}
\CommentTok{\# Ahora predecir con datos de prueba usando el modelo entrenado}
\CommentTok{\# Primero aplicar la transformación del selector de funciones para asegurarse de que se seleccionen las mismas funciones de los datos de prueba}
\NormalTok{X\_test\_filtered }\OperatorTok{=}\NormalTok{ selector.transform(X\_test)}
\NormalTok{prediction\_xgb }\OperatorTok{=}\NormalTok{ xgb\_model.predict(X\_test\_filtered)}
\end{Highlighting}
\end{Shaded}

\hypertarget{precisiuxf3n}{%
\subsection{Precisión}\label{precisiuxf3n}}

\begin{Shaded}
\begin{Highlighting}[]
\NormalTok{skl.metrics.accuracy\_score(y\_test, prediction\_xgb)}
\end{Highlighting}
\end{Shaded}


\end{document}
